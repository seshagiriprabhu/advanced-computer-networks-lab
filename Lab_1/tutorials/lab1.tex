% This text is Free and open Open Source.
% It's a part of presentation made by myself.
% It may be used only for academic purpose
% May, 2012
% Author: Seshagiri Prabhu
% Amrita Vishwa Vidyapeethm 
% seshagiriprabhu@gmail.com
% www.seshagiriprabhu.wordpress.com

\documentclass[12pt]{beamer}
\usetheme{Oxygen}
\usepackage{thumbpdf}
\usepackage{wasysym}
\usepackage{ucs}
\usepackage[utf8]{inputenc}
\usepackage{pgf,pgfarrows,pgfnodes,pgfautomata,pgfheaps,pgfshade}
\usepackage{verbatim}
\usepackage{listings}
\usepackage{courier}
\usepackage{caption}
\usepackage{verbatim} 
\usepackage{upquote}
\usepackage{graphics}
\usepackage{latexsym}
\usepackage{fixltx2e}
\usepackage{graphicx}
\usepackage{hyperref}
\usepackage{amssymb}
\usepackage{ragged2e}
\usepackage{amsmath}
\usepackage{mathtools}
\usepackage{framed,lipsum}
\usepackage{pgf}
\usepackage{fmtcount}% http://ctan.org/pkg/fmtcount
\usepackage{algorithm, algpseudocode}
\usepackage{caption}
\captionsetup[algorithm]{font=scriptsize}
\usepackage{etoolbox}
\makeatother
\newcommand\Colorhref[3][cyan]{\href{#2}{\small\color{#1}#3}}
\newcommand\Fontvi{\fontsize{5}{6}\selectfont}
%\renewcommand\tinyv{\@setfontsize\tinyv{4pt}{6}}
%\renewcommand\tiny{\@setfontsize\tiny{4pt}{6}}
\usepackage{color}

\usepackage{xcolor}

\def\SPSB#1#2{\rlap{\textsuperscript{\textcolor{red}{#1}}}\SB{#2}}
\def\SP#1{\textsuperscript{\textcolor{red}{#1}}}
\def\SB#1{\textsubscript{\textcolor{blue}{#1}}}

\usepackage{listings}
\usepackage{color}

\definecolor{mygreen}{rgb}{0,0.6,0}
\definecolor{mygray}{rgb}{0.5,0.5,0.5}
\definecolor{mymauve}{rgb}{0.58,0,0.82}
\lstset{
 breaklines=true, commentstyle=\color{mygreen}, stepnumber=1, tabsize=2, stringstyle=\color{mymauve}, numberstyle=\tiny\color{mygray}, rulecolor=\color{black}, morekeywords={make,mkdir,gcc, all, clean, F, OO}, basicstyle=\tiny\ttfamily,keywordstyle=\color{blue}, emph={CC, CFLAG, EXECS, PROG, OBJS, OBJECTS, PHONY}, emphstyle=\color{red}, emph={[2]\#using,\#define,\#ifdef,\#endif}, emphstyle={[2]\color{blue}}, stringstyle=\color{red}
}
\newenvironment{variableblock}[3]{%
  \setbeamercolor{block body}{#2}
  \setbeamercolor{block title}{#3}
  \begin{block}{#1}}{\end{block}}
\pdfinfo
{
  /Title       (SN 707 Software Protection)
  /Creator     (Seshagiri Prabhu N)
  /Author      (TeX)
}
\title{CS620 Advanced Computer Networks}
\subtitle{Lab 1}
\author{Course Mentor: Snehal Shetty \\ TA: Seshagiri Prabhu}
\institute[Amrita Vishwa Vidyapeetham] % (optional)
{
  \begin{center}
    \tiny
    Amrita Center for Cyber Security\\
	Amritapuri
  \end{center}  
}

\begin{document}
\frame{\titlepage}
\begin{frame}
	\begin{variableblock}{}{bg=orange,fg=black}{bg=green,fg=red}
		\begin{center}
			\vskip5mm
  			{\large Socket Programming in C}
  			\vskip5mm
  			\vskip5mm  			
		\end{center}
	\end{variableblock}
\end{frame}

\begin{frame}
	{\large Contact info}
	\begin{center}
		{\large Snehal Shetty} \\
		{\small \textcolor{blue}{Office}} \\
		{\small \textcolor{black}{103}} \\
		\vskip1mm
		{\small \textcolor{blue}{email}} \\
		{\small \href{mailto:snehal@am.amrita.edu}{snehal@am.amrita.edu}}\\
		\vskip5mm
		{\large Seshagiri Prabhu} \\
		{\small \textcolor{blue}{Office}} \\		
		{\small \textcolor{black}{110}} \\		
		\vskip1mm
		{\small \textcolor{blue}{email}} \\
		{\small \href{mailto:seshagiriprabhu@am.amrita.edu}{seshagiriprabhu@am.amrita.edu}}\\
	\end{center}
\end{frame}

\begin{frame}
  \frametitle{Outline}
  \tableofcontents[section=1,hidesubsections]
\end{frame}

\newcommand{\icon}[1]{\pgfimage[height=1em]{#1}}

%%%%%%%%%%%%%%%%%%%%%%%%%%%%%%%%%%%%%%%%%
%%%%%%%%%% Content starts here %%%%%%%%%%
%%%%%%%%%%%%%%%%%%%%%%%%%%%%%%%%%%%%%%%%%

\section{Socket}
\begin{frame}
	\frametitle{Sockets}
	{\small Sockets are a protocol independent method of creating a connection between processes. Sockets can be either}
	\begin{itemize}
		\item \textcolor{red}{connection based} or \textcolor{red}{connectionless}: Is a connection established before communication or does each packet describe the destination?
		\item \textcolor{red}{packet based} or \textcolor{red}{streams based}: Are there message boundaries or is it one stream?
		\item \textcolor{red}{reliable} or \textcolor{red}{unreliable}. Can messages be lost, duplicated, reordered, or corrupted?
	\end{itemize}
\end{frame}

%\begin{frame}
%	\frametitle{Socket Characteristics}
%	{\small Socket are characterized by their domain, type and transport protocol. Common domains are:}
%	\begin{itemize}
%		\item \color{blue}{AF_UNIX}:address format is UNIX pathname	
%	\end{itemize}
%\end{frame}
\begin{frame}
	\frametitle{Socket Characteristics}
	{\small Socket are characterized by their domain, type and transport protocol. Common domains are:}
	\begin{itemize}
		\item \textcolor{blue}{\texttt{AF\_UNIX}}: address format is UNIX pathname
		\item \textcolor{blue}{\texttt{AF\_INET}}: address format is host and port number
	\end{itemize}
	\vskip4mm
	{\small Common types are:}
	\begin{itemize}
		\item \textcolor{blue}{Virtual Circuits}: received in order transmitted and reliably
		\item \textcolor{blue}{datagram}: arbitrary order, unreliable
	\end{itemize}
\end{frame}

\begin{frame}
	\frametitle{Socket Characteristics (cont'd)}
	{\small Each socket type has one or more protocols. Ex:}
	\begin{itemize}
		\item TCP/IP (virtual circuits)
		\item UDP (datagram)
	\end{itemize}
	\vskip4mm
	{\small Use of sockets:}
	\begin{itemize}
		\item Connection–based sockets communicate client-server: the server waits for a connection from the client.
		\item Connectionless sockets are peer-to-peer: each process is symmetric.
	\end{itemize}
\end{frame}

\begin{frame}
	\frametitle{Socket APIs}
	\begin{itemize}
		\item \textcolor{blue}{\texttt{socket}}: creates a socket of a given domain, type, protocol
(buy a phone)
		\item \textcolor{blue}{\texttt{listen}}: assigns a name to the socket (get a telephone number)
		\item \textcolor{blue}{\texttt{accept}}: specifies the number of pending connections that can be queued for a server socket. (call waiting allowance)
		\item \textcolor{blue}{\texttt{connect}}: client requests a connection request to a server
(call)	
		\item \textcolor{blue}{\texttt{send, sendto}}: write to connection (speak)
		\item \textcolor{blue}{\texttt{recv, recvfrom}}: read from connection (listen)		
		\item \textcolor{blue}{\texttt{shutdown}}: end the call	
	\end{itemize}
\end{frame}


\section{Problem description}
\begin{frame}[fragile]
	\frametitle{Problem Description}
	\framesubtitle{Compile the program}
	\tiny
	\begin{lstlisting}[language=make, breaklines=true, commentstyle=\color{mygreen}, frame=shadowbox, rulecolor=\color{black}, numbers=left,  numbersep=2pt, numberstyle=\tiny\color{mygray}]
	# Make file
	CC=gcc
	CFLAG= -c -Wall

	EXECS := circle
	PROG  := circle.c
	OBJS  := $(addprefix OBJECTS/, $(addsuffix .o, $(EXECS)))

	all     :$(EXECS)

	OBJECTS :
    		mkdir -p $@

	$(OBJS) :| OBJECTS

	$(EXECS):$(OBJS)
            $(CC) $^ -o $@

	OBJECTS/circle.o :$(PROG)
                $(CC) $(CFLAG) $^ -o $@

	clean   :
    		rm -rf $(EXECS) OBJECTS

	.PHONY  : clean all
	\end{lstlisting}			
\end{frame}
\begin{frame}[fragile]
	\frametitle{Problem Description}
	\framesubtitle{Compile and run the program}
	\tiny
	\begin{block}{Compile the program}
		\begin{lstlisting}[language=bash]
seshagiri@ACCS:~$ make
mkdir -p OBJECTS
gcc -c -Wall circle.c -o OBJECTS/circle.o
gcc OBJECTS/circle.o -o circle
    	\end{lstlisting}
    \end{block}
    
	\begin{block}{Execute the binary file - $circle$}
		\begin{lstlisting}[language=bash]
seshagiri@ACCS:~$ ./circle
0.250
		\end{lstlisting}
	\end{block}
\end{frame}

\begin{frame}
	\frametitle{Problem Description}
	\framesubtitle{What can we understand after executing the binary file?}
	\begin{center}
		\small What can we understand after executing the binary file?
		\small{$<burp>$} \huge {NOTHING!} \small {$</burp>$}
	\end{center}
\end{frame}

\begin{frame}
	\frametitle{Problem Description}
	\framesubtitle{What should we do next?}
	\begin{center}
		\small {What should we do next?} \\
		\large Inspect the source code
	\end{center}
\end{frame}

\begin{frame}[fragile]
	\frametitle{Problem Description}
	\framesubtitle{Intended the code properly for better understanding}
	\begin{lstlisting}[language=C, breaklines=true, commentstyle=\color{mygreen}, rulecolor=\color{black}, numbers=left,  numbersep=2pt, numberstyle=\tiny\color{mygray}] ]
#include <stdio.h>
#define _ -F<00||--F-OO--;

int F=00,OO=00;

main() {
        F_OO();
        printf("%1.3f\n", 4.*-F/OO/OO);
}

F_OO() {
            _-_-_-_
       _-_-_-_-_-_-_-_-_
    _-_-_-_-_-_-_-_-_-_-_-_
  _-_-_-_-_-_-_-_-_-_-_-_-_-_
 _-_-_-_-_-_-_-_-_-_-_-_-_-_-_
 _-_-_-_-_-_-_-_-_-_-_-_-_-_-_
_-_-_-_-_-_-_-_-_-_-_-_-_-_-_-_
_-_-_-_-_-_-_-_-_-_-_-_-_-_-_-_
_-_-_-_-_-_-_-_-_-_-_-_-_-_-_-_
_-_-_-_-_-_-_-_-_-_-_-_-_-_-_-_
 _-_-_-_-_-_-_-_-_-_-_-_-_-_-_
 _-_-_-_-_-_-_-_-_-_-_-_-_-_-_
  _-_-_-_-_-_-_-_-_-_-_-_-_-_
    _-_-_-_-_-_-_-_-_-_-_-_
        _-_-_-_-_-_-_-_
            _-_-_-_
}
	\end{lstlisting}

\end{frame}

\section{Problem Dissection}
\begin{frame}[fragile]
	\frametitle {Problem Dissection}
	\framesubtitle {Macro... Macro..}
	\begin{block}{Unnoticed macro $\_$ }
		\begin{lstlisting}[language=C]
			#define _ -F<00||--F-OO--;
		\end{lstlisting}	
	\end{block}
	\begin{block}{What macro does?}
		\begin{itemize}
			\item Replaces '\texttt{\_}' with {\tiny '\texttt{-F<00||-{}-F-OO-{}-;}'}
			\item for eg: {\tiny '\texttt{\_-\_-\_}'} becomes {\tiny '\texttt{-F<00||-{}-F-OO-{}-;- -F<00||-{}-F-OO-{}-;- -F<00||-{}-F-OO-{}-;}'}
			\item One of the important property of \texttt{||} operator is that if the first condition is \textcolor{red}{true}, it will \textcolor{red}{NOT} check for the second condition unlike \texttt{\&\&} operator.
		\end{itemize}
	\end{block}
\end{frame}

\begin{frame}[fragile]
	\frametitle{Problem Dissection}
	\framesubtitle{Values of variables inside \texttt{F\_OO()}}
	\begin{block}{Values of \texttt{F} and \texttt{OO}}
		\begin{itemize}
			\item \texttt{F} and \texttt{OO} are initially set as $zero$. 
			\item In the first line of \texttt{F\_OO} function, \texttt{-F<00} becomes \textcolor{red}{false} hence checks for the second condition.
			\item \texttt{-{}-F} becomes -1 and \texttt{OO-{}-} remains as 0.
			\item After the execution of the first line, value of F becomes -1 and OO becomes -1.
		\end{itemize}				
	\end{block}
\end{frame}

\begin{frame}[fragile]
	\frametitle{Problem Dissection}
	\framesubtitle{Values of \texttt{F} and \texttt{OO} inside \texttt{F\_OO()}}
	\begin{itemize}
	\item  {\tiny \textbf{Line 1:} texttt{-0<0 (\textcolor{red}{false}) || (-1)-(0--) (\textcolor{green}{executed}); -1<0 (\textcolor{green}{true}) || (-2)-(-1)(\textcolor{red}{not executed}); -1 < 0 (\textcolor{green}{true}) || (-3)-(-2)(\textcolor{red}{not executed})}}. {\tiny \texttt{F = -1, OO = -1}}
	\item {\tiny \textbf{Line 2:} \texttt{1<0 (\textcolor{red}{false}) || (-2)-(1--) (\textcolor{green}{executed}); -2<0 (\textcolor{green}{true}) || (-3)-(-2)(\textcolor{red}{not executed}); -3 < 0 (\textcolor{green}{true}) || (-4)-(-3)(\textcolor{red}{not executed}) -4<0 (\textcolor{green}{true}) || (-5)-(-4)(\textcolor{red}{not executed}); -5 < 0 (\textcolor{green}{true}) || (-6)-(-5)(\textcolor{red}{not executed})}}. {\tiny \texttt{F = -2, OO = -2}}
	\\.
	\\.
	\\.
	\item {\tiny \textbf{Line 16:} \texttt{F = -16, OO = -16} }
	\end{itemize}
\end{frame}

\begin{frame}[fragile]
	\frametitle{Problem Dissection}
	\framesubtitle{Lets go back to \texttt{main()} where \texttt{F\_OO()} was called}
	\begin{block}{Inside \texttt{main()}}
		\begin{lstlisting}[language=C]
			printf("%1.3f\n",4.*-F/OO/OO);
		\end{lstlisting}
		This line calculates the value 0.250 
	\end{block}
\end{frame}

\begin{frame}
	\frametitle{Problem Dissection}
	\framesubtitle{What was the learning from this program?}
	\begin{center}
		\small {What was the learning by inspecting the code} \\
		\huge {NIL}
	\end{center}
\end{frame}

\begin{frame}
	\frametitle{Problem Dissection}
	\framesubtitle{What should be done next?}
	\begin{center}
		\small {What should be done next?} \\
		\small {Lets} \\
	\end{center}
\end{frame}

\begin{frame}
	\frametitle{Problem Dissection}
	\framesubtitle{What should be done next?}
	\begin{center}
		\small {Bingo! Wiki rocks} \\
		{\tiny \url{http://en.wikipedia.org/wiki/International_Obfuscated_C_Code_Contest\#Examples}}
		{\tiny The same example is mentioned in the above mentioned link and it says that the program works by calculating its own area and diameter, and then doing a division to approximate pi!!} \\
		{\small They call it as an Obfuscated C code}
	\end{center}
\end{frame}

\begin{frame}
	\frametitle{Problem Dissection}
	\framesubtitle{How does it calculates the value of $\Pi$?}
	\begin{block}{Lets do reverse engineering!}
		Using the hint in Wiki, we found that program works by calculating its own area and diameter. So we need to make the $\Pi$, Area, radius/diameter looks like \texttt{4.*-F/OO/OO}. \\ So lets do reverse Engineering!
	\end{block}
\end{frame}

\begin{frame}
	\frametitle{Problem Dissection}
	\framesubtitle{Calculating the value of $\Pi$ }
	\begin{block}{Calculating the value of $\Pi$ from Area and radius/diameter}
		Area = $\Pi$ * $radius^{2}$ \\
		$\Pi$ = Area / $radius^{2}$ \\
		As we need 4 in the numerator we may have to substitute radius with diameter.\\
		$\Pi$ = Area / $(diameter/2)^{2}$ \\
		$\Pi$ = 4 * Area / (diameter * diameter) \\
		$\Pi$ = 4 * Area / diameter / diameter \\
	\end{block}
\end{frame}	

\section{Inference}
\begin{frame}
	\frametitle{Inference}
	\framesubtitle{What can we conclude from this?}
	\begin{block}{Conclusion}
		Variable \textcolor{red}{F} is the area of the circle used in the program \\
		The Diameter of the circle is the variable \textcolor{red}{OO}
	\end{block}
\end{frame}

\begin{frame}
	\frametitle{Learning}
	\framesubtitle{What's new I have learned from this?}
	\begin{block}{Learning something new}
		Previously, I have tried to de-obfuscating PHP code (in which they used base64 encoding, rot 13 transformation etc) files during Capture The Flag contests. This is the first time I am hearing about C code obfuscation. That was a nice learning experience!
	\end{block}
\end{frame}

\frame{
  	\vspace{2cm}
	\begin{center}
  		{\huge Thank you!}
	\end{center}
  	\vspace{1cm}
  	\begin{flushright}
        Seshagiri Prabhu \\
    	\structure{\footnotesize{seshagiriprabhu@gmail.com} \\ AM.EN.P2CSN12028 \\  \href{https://github.com/seshagiriprabhu/software-protection-lab}{Assignment repo}}
  	\end{flushright}
}

\end{document}